\documentclass[a4paper,11pt]{article}
\usepackage{amsmath}
\usepackage{amssymb}
\usepackage{geometry}
\usepackage{setspace}
\geometry{left=2cm, right=2cm, top=2cm, bottom=2cm}

\title{\textbf{COMP2711} Homework3}
\author{LIU, Jianmeng 20760163}
\date{}

\begin{document}
    \maketitle

    \begin{spacing}{1.4}

    \setlength{\parindent}{0px}

    \textbf{Question 1:}

    From $32b-21a=19$, we have:
    \begin{align*}
        32b-21a&\equiv 19 (\rm mod \ 21)\\
        32b&\equiv 19 (\rm mod \ 21)\ \ \ (*)
    \end{align*}
    To solve this congruence, we need to find a multiple inverse
    of 32 modulo 21. Notice that gcd$(32, 21)=1$, so by
    using the Euclidean algorithm, we have:
    \begin{align*}
        32 &= 1\cdot 21 + 11\\
        21 &= 1\cdot 11 + 10\\
        11 &= 1\cdot 10 + 1
    \end{align*}
    Reverse the steps(extended Euclidean algorithm), we have:
    \begin{align*}
        1&=11-1\cdot 10\\
         &=11-1\cdot (21-1\cdot 11)\\
         &=2\cdot 11-1\cdot 21\\
         &=2\cdot (32-1\cdot 21)-1\cdot 21\\
         &=2\cdot 32-3\cdot 21
    \end{align*}
    Thus,
    \begin{align*}
        2\cdot 32-3\cdot 21 &\equiv 1 ({\rm mod}\ 21)\\
        2\cdot 32 &\equiv 1 ({\rm mod}\ 21)
    \end{align*}
    which means 2 is a multiple inverse of 32 modulo 21.
    To solve $(*)$, we multiply 2 on both sides,
    \begin{align*}
        2\cdot 32b &\equiv 2\cdot 19 ({\rm mod}\ 21)\\
        b &\equiv 38({\rm mod}\ 21) \equiv 17({\rm mod}\ 21)
    \end{align*}
    Thus, $b=17+21k$, where $k\in \mathbb{Z}$. 
    Since $b\in \mathbb{Z}_{42}$, only $k=0$ and $k=1$ are valid,
    which gives $b=17$ or $b=38$.
    \begin{itemize}
        \item When $b=17$, $32\cdot 17-21a=19$, 
        we get $a=25\in \mathbb{Z}_{42}$.
        \item When $b=38$, $32\cdot 38-21a=19$, 
        we get $a=57\notin \mathbb{Z}_{42}$.
    \end{itemize}

    Therefore, there exists only one pair of $a,b$,
    where $a=25, b=17$.

    \vspace{20pt}

    \textbf{Question 2:}

    \begin{center}
        \begin{tabular}{c|c|c|c|c|c|c|c|c|c|c|c|c}
            \hline
            A&B&C&D&E&F&G&H&I&J&K&L&M\\\hline
            0&1&2&3&4&5&6&7&8&9&10&11&12\\
            \hline\hline
            N&O&P&Q&R&S&T&U&V&W&X&Y&Z\\\hline
            13&14&15&16&17&18&19&20&21&22&23&24&25\\\hline
        \end{tabular}
    \end{center}
    
    (a)From the table above, the original message in integers is:

    \hspace{2em} \texttt{12,4,4,19, ,0,19, ,19,7,4, ,15,0,17,10}

    (b)According to (a), calculate $f(p)$ for each number $p$,
    we can easily get the ciphertext in integers:
    
    \hspace{2em} \texttt{16,2,2,25, ,8,25, ,25,17,2, ,5,8,15,6}

    (c) According to (b) and the table above:

    \hspace{2em} \texttt{QCCZ IZ ZRB FIPG}

    (d) Since $f(p)=(5p+8)\ {\rm mod\ 26}$, we have
    $5p \equiv f(p)-8 {\rm (mod\ 26)}$.
    By extended Euclidean algorithm, it's not difficult to find
    that a multiple inverse of 5 modulo 26 is 21.

    Thus, $p\equiv 21\cdot [f(p)-8]\equiv 21\cdot f(p)+14\
    {\rm (mod\ 26)}$.

    Therefore, $g(c)=(21c+14)\ {\rm mod\ 26}$.


    \vspace{20pt}

    \textbf{Question 3:}

    We first let $m=9\cdot 14\cdot 5=630$,
    $M_1=m/9=70, M_2=m/14=45, M_3=m/5=126$.

    By using extended Euclidean algorithm, we know:

    \hspace{1em} 4 is an inverse of $M_1$ modulo 9, since 
    $4\cdot 70\equiv 4\cdot 7\equiv 1 {\rm (mod\ 9)}$

    \hspace{1em} 5 is an inverse of $M_2$ modulo 14, since 
    $5\cdot 45\equiv 5\cdot 3\equiv 1{\rm (mod\ 14)}$

    \hspace{1em} 1 is an inverse of $M_3$ modulo 5, since 
    $1\cdot 126\equiv 1\cdot 1\equiv 1{\rm (mod\ 5)}$

    So the solutions to the system are those $x$ such that:
    \begin{align*}
        x&\equiv 4\cdot 70\cdot 4+ 8\cdot 45\cdot 5
        + 3 \cdot 126 \cdot 1\\
        &=3298\\
        &\equiv 148{\rm (mod\ 630)}
    \end{align*}
    Therefore, the solutions are those $x$ such that
    $x\equiv 148{\rm (mod\ 630)}$, which can also be written as
    $x=148+630k, k\in \mathbb{Z}$.



    % \vspace{20pt}
    \newpage
    \textbf{Question 4:}

    Note that $1027_{10}=2^{10}+2^1+2^0=(100\ 0000\ 0011)_2$,
    compute:
    \begin{align*}
        8^{2^0}&\equiv 8 {\rm (mod\ 22)}\\
        8^{2^1}&\equiv (8^2)\equiv 20 {\rm (mod\ 22)}\\
        8^{2^2}&\equiv (20^2)\equiv 4 {\rm (mod\ 22)}\\
        8^{2^3}&\equiv (4^2)\equiv 16 {\rm (mod\ 22)}\\
        8^{2^4}&\equiv (16^2)\equiv 14 {\rm (mod\ 22)}\\
        8^{2^5}&\equiv (14^2)\equiv 20 {\rm (mod\ 22)}\\
        8^{2^6}&\equiv (20^2)\equiv 4 {\rm (mod\ 22)}\\
        8^{2^7}&\equiv (4^2)\equiv 16 {\rm (mod\ 22)}\\
        8^{2^8}&\equiv (16^2)\equiv 14 {\rm (mod\ 22)}\\
        8^{2^9}&\equiv (14^2)\equiv 20 {\rm (mod\ 22)}\\
        8^{2^{10}}&\equiv (20^2)\equiv 4 {\rm (mod\ 22)}
    \end{align*}
    According to repeated squaring method, we know that
    \begin{align*}
        8^{1027}&=8^{2^{10}}\cdot 8^{2^1}\cdot 8^{2^0}\\
        &\equiv 4\cdot 20\cdot 8 {\rm (mod\ 22)}\\
        &\equiv 2 {\rm (mod\ 22)}
    \end{align*}
    Therefore, $8^{1027}\equiv 2 {\rm (mod\ 22)}$
    

    \vspace{20pt}

    \textbf{Question 5:}

    To eliminate $y$, we multiply the first congruence by 15,
    the second by 18:
    \begin{align*}
        \left\{
            \begin{array}{l}
                315x+270y\equiv 195\equiv 58 {\rm (mod\ 137)}\ \ \ (1)\\
                576x+270y\equiv 162\equiv 25 {\rm (mod\ 137)}\ \ \ (2)
            \end{array}
        \right.
    \end{align*}
    $(2)-(1)$, we get:
    $$261x\equiv -33 {\rm (mod\ 137)}$$
    Factorize both sides, we get:
    $$3^2\cdot 29x\equiv -3\cdot 11 {\rm (mod\ 137)}\ \ \ (*)$$
    As $116\cdot 13=2^2\cdot 13\cdot 29\equiv 1 {\rm (mod\ 137)}$,
    we know that a multiple inverse of 29 modulo 137 is $2^2\cdot 13$.
    Multiple both sides of $(*)$ by $2^2\cdot 13$, we get:
    \begin{align*}
        3^2\cdot (2^2\cdot 13\cdot 29)x &\equiv 
        -2^2\cdot 3\cdot 11\cdot 13 {\rm (mod\ 137)}\\
        3^2\cdot x &\equiv 65 {\rm (mod\ 137)}\ \ \ (**)
    \end{align*}
    As $99\cdot 18=2\cdot 3^4\cdot 11\equiv 1 {\rm (mod\ 137)}$,
    we know that a multiple inverse of $3^2$ modulo 137 is 
    $2\cdot 3^2\cdot 11$.
    Multiple both sides of $(**)$ by $2\cdot 3^2\cdot 11$, we get:
    \begin{align*}
        3^2\cdot 2\cdot 3^2\cdot 11\cdot x &\equiv 
        65\cdot 2\cdot 3^2\cdot 11 {\rm (mod\ 137)}\\
        x &\equiv 129 {\rm (mod\ 137)}
    \end{align*}
    Since $0\le x\le 136$, $x=129$ is the only solution.

    Bring $x=129$ back to $21x+18y\equiv 13 {\rm (mod\ 137)}$,
    we get:
    \begin{align*}
        18\cdot y &\equiv 44 {\rm (mod\ 137)}\\
        2\cdot 3^2\cdot y &\equiv 44 {\rm (mod\ 137)}
    \end{align*}

    Apply the similar method while finding $x$,
    as $99\cdot 18=2\cdot 3^4\cdot 11\equiv 1 {\rm (mod\ 137)}$,
    we know that a multiple inverse of $2\cdot 3^2$ modulo 137 is 
    $3^2\cdot 11$, multiple it on both sides:
    \begin{align*}
        2\cdot 3^2\cdot 3^2\cdot 11\cdot y &\equiv 
        44\cdot 3^2\cdot 11 {\rm (mod\ 137)}\\
        y &\equiv 109 {\rm (mod\ 137)}
    \end{align*}
    Since $0\le y\le 136$, $y=109$ is the only solution.

    Therefore, the system has only one solution,
    $x=129, y=109$.
    

    \end{spacing}
\end{document}