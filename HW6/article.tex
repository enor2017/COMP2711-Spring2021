\documentclass[a4paper,11pt]{article}
\usepackage{amsmath}
\usepackage{amssymb}
\usepackage{geometry}
\usepackage{setspace}
\usepackage{tcolorbox}
\usepackage{listings}
\geometry{left=2.3cm, right=2.3cm, top=2.4cm, bottom=2.54cm}

\title{\textbf{COMP2711} Homework6}
\author{LIU, Jianmeng 20760163}
\date{}

\begin{document}
    \maketitle

    \begin{spacing}{1.4}

    \setlength{\parindent}{0em}

    \textbf{Question 1:}

    \hspace{2em}
    Assume $A:$ the string contains at least two consecutive 0s, 
    $B:$ the second bit is a 1. Then we want to find $P(A|B).$

    \hspace{2em}
    We first count number of strings that satisfy $A\cap B$.
    The first bit can always be 0 or 1 (since it doesn't effect
    the number of strings that contains consecutive 0s), we only
    need to focus on the last 6 bits, and count the number of strings
    $of\ length\ 6$ that contains at least two consecutive 0s.

    \hspace{2em}
    For the last 6 bits, we count how many strings doesn't contain
    consecutive 0s instead:
    if there are zero 0 and six 1s, 
    there is only one possible string (111111); if one 0 and 
    five 1s, then there are ${6\choose 1}$ strings; 
    if two 0s and four 1s, we insert 0s into the five spaces between
    four 1s(including left and right sides), then there are 
    ${5\choose 2}$ strings; and if three 0s and three 1s,
    there are ${4\choose 3}$ strings; if there are four or more 0s,
    then it must contain consecutive 0s. So in total, there are
    $1+{6\choose 1}+{5\choose 2}+{4\choose 3}=21$ strings of length 6
    that $don't\ contain$ consecutive 0s. Then,
    $$P(A\cap B)=2\cdot \frac{2^6-21}{2^8}=\frac{86}{256}$$
    
    We also now:
    $$P(B)=\frac{2^7}{2^8}=\frac{128}{256}$$

    Thus,
    $$P(A|B)=\frac{P(A\cap B)}{P(B)}=\frac{86}{128}=\frac{43}{64}.$$


    \textbf{Question 2:}

    \hspace{2em}
    Assume $A:$ one is infected with HIV, $B:$ one is tested positive.
    Then we know $P(A)=8\%,\ P(B|A)=98\%,\ P(B|\overline{A})=3\%$.

    (a)
    \begin{align*}
        P(A|B)&=\frac{P(B|A)P(A)}
            {P(B|A)P(A)+P(B|\overline{A})P(\overline{A})}\\
            &=\frac{98\%\cdot 8\%}{98\%\cdot 8\%+3\%\cdot 92\%}\\
            &=73.96\%
    \end{align*} 
    (b)
    $$P(\overline{A}|B)=1-P(A|B)=26.04\%$$
    % \begin{align*}
    %     P(\overline{A}|B)&=\frac{P(B|\overline{A})P(\overline{A})}
    %         {P(B|\overline{A})P(\overline{A})
    %         +P(B|A)P(A)}\\
    %         &=\frac{3\%\cdot 92\%}{3\%\cdot 92\%+98\%\cdot 8\%}\\
    %         &=26.04\%
    % \end{align*} 


    \textbf{Question 3:}

    \hspace{2em}
    (a) We first count how many possiblities are there when 
    $X=k$ ($1\le k\le 6, k\in \mathbb{Z}$). 
    Since $\min\{ D_1,D_2 \}=k$, either $D_1$ or $D_2$ 
    should be $k$, and the other one can be any number 
    $no\ less\ than\ k$, so there are
    $$1\cdot (7-k) + (7-k) \cdot 1 -1=13-2k$$ different possibilities.
    Note here we minus $1$ because $D_1=D_2=k$ should be counted
    only once. Thus,
    $$P(X=k)=\frac{13-2k}{6\cdot 6}=\frac{13-2k}{36}.$$
    The distribution function of $X$ is:
    $$f(X)=\frac{13-2k}{36},\ (1\le X\le 6, X\in \mathbb{Z})$$

    
    (b) We can list all possible values of $X$ and their corresponding
    possibilities $P(X)$:

    \begin{center}
        \begin{tabular}{c|c|c|c|c|c|c}
            \hline
            $X$ & 1 & 2 & 3 & 4 & 5 & 6\\\hline
            $P(X)$ & $\frac{11}{36}$ & $\frac{9}{36}$ & $\frac{7}{36}$&
            $\frac{5}{36}$ & $\frac{3}{36}$ & $\frac{1}{36}$\\\hline 
        \end{tabular}
    \end{center}

    Therefore, $\displaystyle
    E(X)=1\cdot \frac{11}{36}+2\cdot \frac{9}{36} +
    3\cdot \frac{7}{36} +4\cdot \frac{5}{36} 
    +5\cdot \frac{3}{36}+6\cdot \frac{1}{36}=\frac{91}{36}.$

    \vspace{10pt}

    (c) Firstly, we find $E(X^2)$:

    \begin{center}
        \begin{tabular}{c|c|c|c|c|c|c}
            \hline
            $X^2$ & 1 & 4 & 9 & 16 & 25 & 36\\\hline
            $P(X^2)$ & $\frac{11}{36}$ & $\frac{9}{36}$ & $\frac{7}{36}$&
            $\frac{5}{36}$ & $\frac{3}{36}$ & $\frac{1}{36}$\\\hline 
        \end{tabular}
    \end{center}
   
    Therefore, $\displaystyle
    E(X^2)=1\cdot \frac{11}{36}+4\cdot \frac{9}{36} +
    9\cdot \frac{7}{36} +16\cdot \frac{5}{36} 
    +25\cdot \frac{3}{36}+36\cdot \frac{1}{36}=\frac{301}{36}.$

    $$V(X)=E(X^2)-E(X)^2=\frac{301}{36}-\left( \frac{91}{36}\right)^2
    =\frac{2555}{1296}$$


    \vspace{20pt}

    \textbf{Question 4:}

    \hspace{2em}
    Firstly, assume a directed multigraph having no isolated vertices
    has an Euler circuit. Since there exists a path from any vertex
    to any other vertex, the graph must be strongly connected, 
    so it must also be weakly connected. If we follow the Euler circuit,
    when the circuit passes a vertex, it adds one in-degree and 
    one out-degree to this vertex(when circuit passes the vertex,
    it comes into the vertex and then goes out of it). Thus 
    the in-degree and out-degree of each vertex are equal.

    \hspace{2em}
    Conversely, if a graph is weakly connected and the in-degree
    and out-degree of each vertex are equal, we can use the 
    algorithm provided during lecture to construct an Euler circuit:

    \newpage
    \begin{tcolorbox}
        \begin{lstlisting}[language=Python]
    c := empty cycle
    while there are still edge not taken yet
        u := any vertex already seen
        c2 := Find-Cycle(u)
        insert c2 into c at u

    Find-Cycle(u):
    while u has an edge not taken yet
        take that edge {u,v}
        u := v
        \end{lstlisting}
    \end{tcolorbox}

    \hspace{2em}
    In the algorithm above, we use \verb|Find-Cycle(u)| to 
    find a cycle. Since all degrees are even(in-degree and out-degree 
    of each vertex are equal), cycle must exist.
    However, it may not find a cycle that contains all edges, 
    so we repeat executing the \verb|Find-Cycle(u)|, and 
    connect all cycles into one. Then appearently, 
    the final result will form a Euler cycle.

    \hspace{2em}
    In conclusion, a directed multigraph having no isolated vertices
    has an Euler circuit iff the graph is weakly connected and the 
    in-degree and out-degree of each vertex are equal.


    \end{spacing}
\end{document}