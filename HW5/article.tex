\documentclass[a4paper,11pt]{article}
\usepackage{amsmath}
\usepackage{amssymb}
\usepackage{geometry}
\usepackage{setspace}
\geometry{left=2.3cm, right=2.3cm, top=2.4cm, bottom=2.54cm}

\title{\textbf{COMP2711} Homework5}
\author{LIU, Jianmeng 20760163}
\date{}

\begin{document}
    \maketitle

    \begin{spacing}{1.4}

    \setlength{\parindent}{0em}

    \textbf{Question 1:}

    (a) There are $5+11=16$ teachers and students in total. When they
    sit in a row, they provide 17 spaces(represent by $\circ$):
    $$\circ 1 \circ 2 \circ 3 \circ \cdots \circ 15 \circ 16 \circ$$
    Firstly, choose 4 spaces from those 17 spaces where guests can seat:
    ${17\choose 4}$, then, guests can swap their positions and teachers and
    students can also swap their positions, this will be $4!\cdot 16!$.
    So the answer is:
    $${17\choose 4}\cdot 4!\cdot 16!$$

    \vspace{5pt}

    (b) There are $5+11=16$ teachers and students in total. When they
    sit in a row, they provide 16 spaces. 
    Firstly, choose 4 spaces from those 16 spaces where guests can seat:
    ${16\choose 4}$, then, guests can swap their positions, which will 
    be $4!$, and teachers and students can also swap their positions, 
    but this time it will be $\frac{16!}{16}$, since arranged in circle.
    So the answer is:
    $${16\choose 4}\cdot 4!\cdot \frac{16!}{16}={16\choose 4}\cdot 4!\cdot 15!$$

    \vspace{5pt}

    (c) There are $5+4=9$ teachers and guests in total. When they
    sit in a row, they provide 10 spaces. Since there are 11 students,
    it's impossible to let them sit in 10 spaces, i.e., 
    ${10\choose 11}=0$. Thus the answer is 0.

    
    \vspace{20pt}

    \textbf{Question 2:}

    \hspace{2em}
    Note that the passcode follows non-decreasing order, so as long as
    we pick out 30 alphabets, there is only one possible order
    to arrange them. Since $2+7+1+1=11$ alphabets have already been 
    chosen, we only need to pick $30-11=19$ alphabets from A $\sim $ Z.

    % \hspace{2em}
    % If those 19 alphabets only contain one kind of letter,(e.g. 
    % all choose A/ all choose B, ...) then it's trivial that
    % there are ${26\choose 1}$ ways.

    % \hspace{2em}
    % If those 19 alphabets contain two kinds of letter,(e.g. only contains
    % A and B, ... ) then we consider it in two steps: firstly choose 2 kinds
    % of letter from 26 letters, ${26\choose 2}$ ways, then we consider
    % how many of them will we choose, respectively. 
    % This can be easily solved by thinking of putting 19 balls into 2 boxes,
    % (where the boxes must be non-empty) so there are ${18\choose 1}$ ways.
    % Thus, there are ${26\choose 2}\cdot {18\choose 1}$ in total.

    % \hspace{2em}
    % By the same method, if those 19 alphabets contain $k$ different letters,
    % there are ${26\choose k}\cdot {18\choose k-1}$ ways in total, where
    % $1\le k\le 19$.

    % \hspace{2em}
    % To sum up, there are:
    % $$\sum_{k=1}^{19}{26\choose k}\cdot {18\choose k-1}$$
    % possible passwords.

    \hspace{2em}
    To solve this problem, we can consider there are 19 balls and 
    25 bars arranged in a row, such that the 25 bars separate balls into
    26 groups, and the number of balls in group $i$ represents
    the number of the $i$-th alphabet in the passcode. For example, 
    if we use ``$\circ$'' to denote balls and use ``$|$'' to denote bars, then
    $$|\circ \circ | \circ | \circ \circ \circ | \cdots$$
    means zero A, two B, one C, three D... are chosen.

    \hspace{2em}
    Thus, we only need to choose 19 positions for balls among the
    total $19+25=44$ items, which gives ${44\choose 19}$ ways. 
    That means we have ${44\choose 19}$ ways to pick 19 alphabets 
    from A $\sim $ Z, and also means there are this number of 
    different passcodes.

    \hspace{2em}
    The final answer is ${44\choose 19}$.


    \vspace{20pt}


    \textbf{Question 3:}

    \hspace{2em}
    Assume that there are $n-3$ apples and $n-2$ pears in total,
    and you would like to pick 10 fruits to eat. So we would like
    to calculate how many ways can you do that in two methods:
 
    \textbf{Method 1:} 
    
    \hspace{2em}
    Since we just want to choose any 10 fruit in 
    $2n-5$ fruites, so there are ${2n-5\choose 10}$ ways.

    \textbf{Method 2:} 
    
    \hspace{2em}
    We enumerate how many apples to choose: 
    for example, we want to choose $k$ apples and $10-k$ pears, out of
    $n-3$ apples and $n-2$ pears, so there are 
    ${n-3\choose k}\cdot {n-2\choose 10-k}$ ways.
    Note that $0\le k \le 10$, so there're 
    $\sum_{k=0}^{10}{n-3\choose k}\cdot {n-2\choose 10-k}$ ways in total.

    Therefore, we have:
    $${2n-5\choose 10}=\sum_{k=0}^{10}{n-3\choose k}\cdot {n-2\choose 10-k}$$


    \vspace{20pt}

    \textbf{Question 4:}

    \hspace{2em}
    We try to simply consider one team as a whole(as one person), 
    then there'll only be 
    19 people sitting in a row. Under this assumption, there are
    ${10\choose 1}\cdot 19!\cdot 2!$ ways.

    \hspace{2em}
    However, we find that some situations have been counted more than once.
    For example, if we denote 10 teams as $1,2,\cdots 10$ and the two
    people inside one team as $A, B$, then the situation:
    $$1_A 1_B 2_A 2_B 3_A 3_B 4_A 4_B\cdots 10_A 10_B$$
    will be counted when we consider team 1 as a whole, and will also be 
    counted when we consider team 2 as a whole, and also 3, 4, $\cdots$ 10
    as a whole.

    \hspace{2em}
    Thus, we need to use Inclusion-Exclusion Principle, and the final result
    will be:
    $$\sum_{k=1}^{10}(-1)^{(k+1)}\cdot {10\choose k}\cdot (20-k)!\cdot (2!)^k$$
    

    

    
    \end{spacing}
\end{document}